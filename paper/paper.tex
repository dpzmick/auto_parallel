\documentclass{article}
\usepackage{minted}
\usemintedstyle{emacs}

\begin{document}

% question: Can macros be used to create interesting and useful parallelism
% tools for recursive functions in clojure.
\section{Introduction}
% most important section in paper:
% - introduce the problem being solved
% - prove that this problem is interesting
% - why is this hard?
% - how does it differ from other approaches?
% - what are key results/limitations
Clojure is a lisp--like language running on the Java Virtual Machine.
The language's default immutable data structures and it's sophisticated STM system make it well suited for parallel and concurrent programming.
Additionally, because Clojure runs on the Java Virtual Machine, Clojure developers can take advantage of existing cross--platform parallelism libraries, such as Java's excellent ExecutorService framework, to write parallel code.

Taking advantage of Clojure's parallel potential is not entirely straightforward, however.
Interfacing with Java libraries directly is simple, but the interface code often feels slightly out of place when surrounded by other Clojure code.
The STM system is easy to use, but the STM constructs are often too low level to be immediately useful.
% for example, writing a parallel map with atoms would be a big pain

% is control flow the right word here? I feel like something else might be more appropriate.
% Recursion is control flow, parlet is control flow, so I guess it makes sense.
As a result, there are a variety of libraries designed to allow developers to take advantage of the parallelism and concurrency potential in the language.
Many of these library functions and builtins are data-parallel; they are designed to apply some sort of operation to a set of data.
This is often the kind of parallelism that developers desire, or are accustomed to identifying, but Clojure's nature as a functional language with immutable structures also makes it possible to exploit parallelism in the control flow of the code.

% even if it is expressed in control flow, it is still data parallelism probably
Using Clojure's macro system, we have implemented a few macros which allow developers to take advantage of Clojure's parallelism potential when their parallelism is more directly expressed in the control flow of their application.
We have shown that it is possible to attain noticeable degrees of parallelism with minimal code changes (with respect to serial code) through these macros, without the need for sophisticated dependency analysis (find OpenMP paper) often required in other parallelism systems.

% the other existing libraries are also a testament to this statement
% we sort of side stepped this by saying that the macros only work on pure code.
% Need to demonstrate that idiotmatic clojure code is actually pure and that these macros actually make sense.

This paper is divided into $n$ sections. The first discusses the macros we've implemented, along with the appropriate use cases for each. In the second section, we present benchmarks demonstrating the cases in which the parallelism macros can (and cannot) improve performance of existing code.

\section{Macros}
\subsection{parlet}
The first of the parallel macros is called parlet.
Parlet is a modified version of the Clojure let form.
The parlet macro has exactly the same semitics as Clojure's let, but it evaluates all of
it's bindings in parallel (if there is no dependencies between them).
For example, suppose I had some long running function foo, if I needed to add
the result of two calls to this function, I could do the following to execute my long running calls in parallel:

% TODO use a figure
\begin{minted}{clojure}
    (parlet
      [a (foo value1)
       b (foo value2)]
       ... ; some other code here
      (+ a b))
\end{minted}

The execution of each of the function calls done in a ForkJoinPool.
The calls to foo both start in the background, then we attempt to evaluate the body of the let.
This means that the code in the body of the let which does not depend on the computations of a and b can execute without delay.
This is obviously only safe when foo does not have side effects which may impact the other code in the function.
In Clojure, it is very common (although not guaranteed by the compiler) for function calls to be pure.

Nested parlet macros will also behave quite well.
For example:

\begin{minted}{clojure}
    (parlet
      [a (foo 100)]
      ;; some other code not using a
      (parlet
        [b (foo 200)]
        (+ a b)))
\end{minted}

Both calls to the function foo will be processed in the background, until we reach the line using the variables that the results are bound to.
This means that parlets can be nested with little concern (again as long as functions are pure).

Code like this may not arise when the code is human generated, but it may arise when the code is generated by another macro.
We will see some examples of this later.

The parlet macro also supports dependency detection.
Clojure let forms can use names defined previously in the let, and the bindings are evaluated from first to last.
The following form is valid (the binding to b used the value defined in the first binding):

\begin{minted}{clojure}
    (let
     [a 1
      b (+ 1 a)]
     a)
\end{minted}

This form would evaluate to 2.
If we plan on evaluating each binding in parallel, we can't allow the bindings to have dependencies.
So, the parlet macro looks through the bindings in the macro to determine if any of them depend on
any of the names defined previously in the macro.
If there are any, the macro will halt the compiler and report an error to the user.

This simple dependency check, as well as the commonplace purity in Clojure code, allow us to ensure correct parallelism with this simple macro.
Because the Clojure compiler does not enforce purity, we still need to trust the programmer to validate the purity assumption before inserting the macro.
% but lots of clojure code is already pure blah blah blah

\subsection{parexpr}

Parexpr breaks down expressions and (aggressively) evaluates them in parallel.
parexpr uses parlet to parallelize forms.
Suppose again that I had a long running function foo, I wanted to call it twice, and add the results.
With parexpr the following will make both calls to foo in parallel:

\begin{minted}{clojure}
    (parexp (+ (foo value1) (foo value2)))
\end{minted}

The macro crawls the expression and expands every subexpression into a parlet.
This macro has limited applicability, but, we can also convince ourselves of its correctness easily.

\subsection{defparfun}

The defparun macro is the most interesting of the 3 macros.
defparfun allows a programmer to parallelize calls to a recursive function.
This is best explained with an example:

% TODO check my own stupid syntax
\begin{minted}{clojure}
    (defparfun fib [n] (> n 30)
      (if (or (= n 1) (= n 2))
          1
          (+
            (fib (- n 1))
            (fib (- n 2)))))
\end{minted}

This defines a parallel Fibonacci function, which will only execute in parallel when the value of it's argument is greater than 30.

% TODO actual explain this
The macro first identifies any recursive call sites, then moves the function call as far ``up'' in the function as it can, without breaking any dependencies.
Again, this dependency check is relatively simple.
After this transformation, it is possible to replace the let forms which bind the function results to their values with parlet forms providing the same bindings.
The introduction of the parlet form introduces parallelism, so each recursive call will execute in the ForkJoin pool, in parallel.
Each of these subsequent recursive calls may create more tasks, which will also be executed in the pool.
ForkJoin pools are designed to handle this type of computation, so the computation will proceed without blocking (as long as function is pure and performs no thread--blocking I/O).
% TODO reference a ForkJoin pool paper

\end{document}
